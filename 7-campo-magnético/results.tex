\section{Resultados}\label{sec:resultados}

Con lo explicado en el apartado~\ref{sec:dispositivo-experimental}, medimos el �ngulo de desviaci�n $\alpha$ de la br�jula
para distintas intensidades $I$ circulando por el cable conductor y con distintas distancias $d$ entre la br�jula y el cable.

La tabla~\ref{tab:angulo} recoge las medidas tomadas.

\begin{table}[tbh!]
    \caption{Tabla de desviaciones de la aguja para diferentes intensidades. Cada serie corresponde a una distancia $d_i$ entre la aguja y el cable.}
    \label{tab:angulo}
    \begin{centering}
        \begin{tabular}{|P{24px}|P{26px}|P{26px}|P{26px}|P{26px}|P{26px}|}
            \hline
            & 5\,cm                   & 10\,cm                  & 15\,cm                  & 20\,cm                  & 25\,cm                                \\
            \hline
            $I$\,(A)  & $\alpha$\,(\textdegree) & $\alpha$\,(\textdegree) & $\alpha$\,(\textdegree)  & $\alpha$\,(\textdegree) & $\alpha$\,(\textdegree)\\
            \hline
            \csvreader[late after line= \\, /csv/separator=semicolon ]{./files/data/angulo_es.csv}{}% use head of csv as column names
            {\csvcoli & \csvcolii               & \csvcoliii              & \csvcoliv               & \csvcolv                & \csvcolvi}% specify your columns here
            \hline
        \end{tabular}
    \end{centering}
\end{table}
