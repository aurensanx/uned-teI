\section{Introducci�n}

La ley de Biot y Savart nos permite calcular el campo magn�tico en un punto $P$ del espacio situado a una distancia $r$ de un conductor rectil�neo.

Las l�neas de campo son circunferencias conc�ntricas contenidas en un plano perpendicular al conductor.
El campo magn�tico $B$ en el punto $P$ tiene direcci�n tangente a las l�neas de campo y m�dulo:
\begin{equation}
    \label{eq:campo}
    B = \frac{\mu_0 I}{2\pi r}
\end{equation}

En la ecuaci�n anterior, $\mu_0$ representa la permeabilidad magn�tica del vac�o y tiene como valor $4\pi \times 10^{-7}\,$T$\cdot$m/A.




