\section{Conclusiones}

El valor del campo magn�tico obtenido en las secciones~\ref{subsec:analisis-1} y~\ref{subsec:analisis-2}
no se aproxima al valor real consultado en la referencia~\cite{1}, aunque s� es del mismo orden de magnitud.
A continuaci�n, intentamos analizar las posibles causas.

Los intervalos de confianza del campo calculado en la secci�n~\ref{subsec:analisis-1} se solapan en su mayor�a.
Por esta raz�n, descartamos que errores aleatorios puedan haber influido en el resultado final.

Nos inclinamos por pensar en que ha habido alg�n error sistem�tico en el montaje experimental que no hemos podido detectar.
Por ejemplo, si el amper�metro marcaba unos valores de intensidad $I$ m�s bajos de los que realmente circulaban por el cable conductor,
el campo inducido $B$ ser�a tambi�n mayor y, para mantener el �ngulo de desviaci�n $\alpha$ constante, la componente horizontal del campo
magn�tico terrestre $H$ tendr�a que disminuir, aproxim�ndose al valor real.



