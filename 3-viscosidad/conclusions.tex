\section{Cuestiones}

\subsection{\label{subsec:c1}}

Debemos calcular $\Delta t$ para el que:
\begin{equation*}
    v_{\infty} - v(\Delta t) < 0.005 \, v_{\infty}
\end{equation*}

Tenemos:
\begin{equation*}
    v(t) = - \alpha\, \beta \, (1 - e^{-\alpha t}) \; \rightarrow \; v_{\infty} = - \alpha\, \beta
\end{equation*}

Sustituyendo:
\begin{equation*}
    - \alpha\, \beta  + \alpha\, \beta \, (1 - e^{-\alpha \Delta t}) < - 0.005\, \alpha \, \beta
\end{equation*}

Esta desigualdad se simplifica a:
\begin{equation*}
    e^{-\alpha \Delta t} > 0.005
\end{equation*}

Tomando logaritmos:
\begin{equation*}
    -\alpha \, \Delta t > \ln{0.005} \; \rightarrow \; \Delta t > \frac{1}{\alpha} \, \ln{200}
\end{equation*}

Sustituyendo el valor de $\alpha$:
\begin{equation*}
    \Delta t > \frac{2 \, r^2 \, \rho_b}{9\, \eta} \, \ln{200}
\end{equation*}

Evaluando con valores $r = 2.42\times 10^{-3}\,$m, $\rho_b = 1000\, \text{kg/m$^3$}$ y $\eta = 0.084\,\text{kg/m$\cdot$s}$,
obtenemos:
\begin{equation*}
    \Delta t > 0.082\,\text{s}
\end{equation*}

\subsection{\label{subsec:c2}}

La densidad del acero es $\rho_{acero} = 7850\,$kg/m$^3$.

En el ap�ndice~\ref{sec:appendix} se adjunta el script de MATLAB usado por tanteo en esta secci�n.

Seg�n nuestros c�lculos, para que una gota de agua tarde $19.8\,$s en recorrer $20\,$cm de aceite, con los valores de $\rho = 920\, \text{kg/m$^3$}$ y
$\eta = 0.084\,\text{kg/m$\cdot$s}$, su radio debe valer $r = 2.2$\,mm.

Con ese mismo radio $r$, y sustituyendo la densidad del agua por la del acero, el tiempo que tardar�a en caer la esfera de acero ser�a
$0.326\,$s.

\subsection{\label{subsec:c3}}

Las ecuaciones que describen este movimiento ascendente son las mismas que hemos usado en la pr�ctica,~\ref{eq:newton} y~\ref{eq:solution}.

La �nica diferencia es que en este caso tendr�amos que $\rho$ es igual a la densidad del agua y $\rho_b$ a la densidad de la gota de aceite.
Esto implica:
\begin{equation*}
    \frac{\rho}{\rho_b} > 1 \; \rightarrow \; \beta < 0,
\end{equation*}
y, por tanto, $z(t)$ es siempre creciente.

El r�gimen de velocidad constante solo se alcanzar�a en el infinito, pero procediendo de la misma manera que en la secci�n~\ref{subsec:c1},
con valores de la densidad de la gota de aceite $\rho_b = 920\, \text{kg/m$^3$}$ y viscosidad del agua
$\eta = 0.001\,\text{kg/m$\cdot$s}$, obtenemos un valor de $\Delta t = 6.4\,$s.

Este valor de $\Delta t$ es mucho mayor que el obtenido para la gota de agua en aceite, y es debido a que $\alpha$ es mucho m�s peque�o
y el t�rmino exponencial adquiere m�s importancia en la ecuaci�n de movimiento.

Utilizando el mismo script~\ref{sec:appendix} que en la secci�n~\ref{subsec:c2}, con los valores de $\rho$, $\rho_b$ y $\eta$ actualizados,
el tiempo $t$ que tardar�a una gota de aceite del mismo radio $r = 2.2$\,mm en subir $20\,$cm en agua ser�a $19.9\,$s.

\subsection{}

La gota de agua, en su ca�da, est� sometida a la fuerza de empuje incluida en la ecuaci�n~\ref{eq:newton}.
Esta fuerza se aplica solo en un hemisferio de la gota.
Por tanto, la presi�n que sufre la gota por el l�quido que la rodea no es igual en todos los puntos de su superficie.

Podr�amos considerar una secci�n de la gota en un plano vertical.
Esta secci�n deber�a ser una elipse, y su excentricidad nos indicar�a su grado
de separaci�n de la esfericidad.

\subsection{}

La presi�n del l�quido sobre la burbuja har�a que esta cambiase su volumen en un factor mucho mayor que la gota de l�quido.

Por tanto, el radio $r$ ya no se podr�a considerar constante y depender�a de la posici�n de la burbuja en el l�quido.
