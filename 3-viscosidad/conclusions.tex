\section{Cuestiones}

\subsection{}

Debemos calcular $\Delta t$ para el que:
\begin{equation*}
    v_{\infty} - v(\Delta t) < 0.005 \, v_{\infty}
\end{equation*}

Tenemos:
\begin{equation*}
    v(t) = - \alpha\, \beta \, (1 - e^{-\alpha t}) \; \rightarrow \; v_{\infty} = - \alpha\, \beta
\end{equation*}

Sustituyendo:
\begin{equation*}
    - \alpha\, \beta  + \alpha\, \beta \, (1 - e^{-\alpha \Delta t}) < - 0.005\, \alpha \, \beta
\end{equation*}

Esta desigualdad se simplifica a:
\begin{equation*}
    e^{-\alpha \Delta t} > 0.005
\end{equation*}

Tomando logaritmos:
\begin{equation*}
    -\alpha \, \Delta t > \ln{0.005} \; \rightarrow \; \Delta t > \frac{1}{\alpha} \, \ln{200}
\end{equation*}

Sustituyendo el valor de $\alpha$:
\begin{equation*}
    \Delta t > \frac{2 \, r^2 \, \rho_b}{9\, \eta} \, \ln{200}
\end{equation*}

Evaluando con valores $r = 2.42\times 10^{-3}\,$m, $\rho_b = 920\, \text{kg/m$^3$}$ y $\eta = 0.084\,\text{kg/m$\cdot$s}$,
obtenemos:
\begin{equation*}
    \Delta t > 0.076\,\text{s}
\end{equation*}

\subsection{}

\subsection{}

\subsection{}

\subsection{}

Volumen de las gotas cambia.