\section{Dispositivo experimental}

Esta pr�ctica est� dividida en 3 experimentos distintos.

En el primero, determinamos el radio de una gota de agua utilizando un cuentagotas y dejando caer un n�mero elevado de gotas ($\approx 100$)
en una jeringa aforada para determinar su volumen total.
Conocido el n�mero de gotas, se calcula el volumen y radio de una sola gota.

En el segundo experimento, montamos una balanza de construcci�n casera a partir de una regla, un soporte y dos vasos de pl�stico iguales,
como indica la figura~\ref{fig:balanza}.


\colorlet{mydarkblue}{blue!50!black}
\colorlet{watercol}{blue!80!cyan!10!white}
\colorlet{darkwatercol}{blue!80!cyan!20!white}
\tikzstyle{water}=[draw=mydarkblue,top color=watercol!90,bottom color=watercol!90!black,shading angle=5]
\tikzstyle{vertical water}=[water,
top color=watercol!90!black!90,bottom color=watercol!90!black!90,middle color=watercol!80,shading angle=90]

\colorlet{mydarkyellow}{olive!50!black}
\colorlet{oilcol}{olive!10!yellow!90!white}
\colorlet{darkoilcol}{olive!20!yellow!80!white}
\tikzstyle{oil}=[draw=mydarkyellow,top color=oilcol!90,bottom color=oilcol!90!black,shading angle=5]
\tikzstyle{vertical oil}=[oil,
top color=oilcol!90!black!90,bottom color=oilcol!90!black!90,middle color=oilcol!80,shading angle=90]

\begin{figure}[tbh]
    \begin{center}
        \begin{tikzpicture}[black!75]

            % Supporting structure
            \fill [pattern = north west lines] (0,0)coordinate(origin) rectangle ++(3,-0.2);
            \draw[thick] (origin) -- ++(3,0);

            \draw[] (1.5,0.2) circle (0.2);
            \draw[] (0,0.4) -- ++(3,0);

            \begin{scope}[shift={(2.6,0.4)}, scale=0.5]
                \def\H{0.9}      % tank horizontal radius
                \def\L{0.8}      % tank vertical radius
%
%                % WATER
                \draw[vertical water]
                (0,0) -- (0,\H)  -- (\L, \H) -- (\L,0) -- (0,0);
                \draw[]
                (\L, \H) -- (\L, 1.2)  -- (0, 1.2) -- (0, \H);
            \end{scope}

            \begin{scope}[shift={(0,0.4)}, scale=0.5]
                \def\H{1}      % tank horizontal radius
                \def\L{0.75}      % tank vertical radius
%
%                % WATER
                \draw[vertical oil]
                (0,0) -- (0,\H)  -- (\L, \H) -- (\L,0) -- (0,0);
                \draw[]
                (\L, \H) -- (\L, 1.2)  -- (0, 1.2) -- (0, \H);
            \end{scope}


        \end{tikzpicture}
        \caption{Balanza de construcci�n casera}
        \label{fig:balanza}
    \end{center}
\end{figure}

El procedimiento consiste en llenar el primer vaso de un volumen conocido de aceite, y el segundo vaso de un volumen tambi�n conocido de agua
pero inferior al que har�a que los pesos fuesen iguales.
Entonces se a�aden gotas de agua al segundo vaso hasta equilibrar la balanza.
Conocido el volumen de una gota de agua, se puede calcular la relaci�n entre los vol�menes y, de all�,
la relaci�n entre las densidades de los l�quidos.

El tercer experimento consiste en dejar caer una gota de agua en una botella llena de aceite y medir el tiempo
que tarda en recorrer dos puntos marcados en la botella.
El punto superior debe estar un poco alejado de la superficie de separaci�n aire-aceite, para que el movimiento
sea lo m�s uniforme posible.
Del tiempo calculado, conocida la distancia entre los dos puntos, determinamos la velocidad de la ca�da de la gota de agua en el aceite.



