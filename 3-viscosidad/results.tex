\section{Procedimiento}\label{sec:resultados}

\subsection{Primer experimento: Determinaci�n del radio de la gota de agua}\label{subsec:primer}

La tabla~\ref{tab:radio} muestra 5 mediciones del volumen para un n�mero de gotas determinado.
El radio $r_i$ se calcula a partir del volumen y dividiendo por el n�mero de gotas.

\begin{table}[h!]
    \caption{Radio $r_i$ de las gotas de agua producidas por el cuentagotas}
    \label{tab:radio}
    \begin{centering}
        \begin{tabular}{|P{60px}|P{68px}|P{64px}|}
            \hline
            n� de gotas & volumen (cm$^3$) & $r_i\,$(mm)                           \\
            \hline
            \csvreader[late after line= \\, /csv/separator=semicolon ]{./files/data/radio.csv}{}% use head of csv as column names
            {\csvcoli   & \csvcolii        & \csvcoliv    }% specify your columns here
            \hline
        \end{tabular}
    \end{centering}
\end{table}

El radio medio de una gota con el error debido a la dispersi�n observada es:
\begin{equation*}
    r = (1.91 \pm 0.02)\, \text{mm}
\end{equation*}

La dispersi�n aleatoria del valor medio de $r$ se ha calculado como:
\begin{equation*}
    \sigma_r = \sqrt{\frac{\sum (r -r_i)^2}{n(n-1)}}
\end{equation*}

Con este dato, determinamos el volumen de una gota de agua con su error asociado mediante las expresiones:
\begin{equation*}
    V = \frac{4}{3}\pi r^3 \hspace{10px}; \hspace{10px}
    \epsilon_V = 4\pi r^2 \epsilon_r
\end{equation*}

El valor del volumen $V$ es:
\begin{equation*}
    V = (29 \pm 2)\, \text{mm}^3
\end{equation*}


\FloatBarrier

\subsection{Segundo experimento: Determinaci�n de la densidad del aceite}

En este experimento llenamos un vaso de pl�stico con $V = 10\,$cm$^3$ de aceite.
A continuaci�n, llenamos el otro vaso de pl�stico con un volumen $V_a$ de agua y a�adimos $n$ gotas hasta que se equilibra la balanza.
Utilizamos el valor del volumen de una gota de agua calculado en el apartado~\ref{subsec:primer} para determinar el volumen
total de agua $V_T$ a partir de $n$.
Por �ltimo, la relaci�n entre el volumen del aceite $V$ y el volumen del agua $V_T$ nos permite hallar la relaci�n entre
las densidades del aceite y del agua $\rho / \rho_b$.

La tabla~\ref{tab:densidad} muestra las medidas y resultados.

\begin{table}[h!]
    \caption{Volumen de aceite frente a volumen del agua que equilibra la balanza}
    \label{tab:densidad}
    \begin{centering}
        \begin{tabular}{|P{36px}|P{36px}|P{24px}|P{36px}|P{36px}|}
            \hline
            $V\,(\text{cm}^3)$ & $V_a\,(\text{cm}^3)$ & $n$        & $V_T\,(\text{cm}^3)$ & $(\rho/\rho_b)_i$                     \\
            \hline
            \csvreader[late after line= \\, /csv/separator=semicolon ]{./files/data/densidad.csv}{}% use head of csv as column names
            {\csvcoli          & \csvcolii            & \csvcoliii & \csvcoliv            & \csvcolvi  }% specify your columns here
            \hline
        \end{tabular}
    \end{centering}
\end{table}

Utilizando la misma expresi�n de la dispersi�n que en la secci�n~\ref{subsec:primer}
para determinar el error asociado al valor medio de la relaci�n entre densidades, obtenemos un
resultado de:
\begin{equation*}
    \frac{\rho}{\rho_b} = 0.986 \pm 0.006
\end{equation*}

Recordando que el valor de la densidad del agua en el Sistema Internacional de unidades es $\rho_b = 1000\, \text{kg/m}^3$,
obtenemos el siguiente resultado para la densidad del aceite:

\begin{equation*}
    \rho = (986 \pm 6)\, \text{kg/m$^3$}
\end{equation*}

\subsection{Tercer experimento: Determinaci�n de la velocidad de ca�da de la gota de agua}

\subsection{C�lculo de la viscosidad del aceite}

\begin{align*}
    \epsilon_{\eta} = & \biggl| \frac{2}{9} \, \frac{g\,r^2}{v} \biggr|  \, \epsilon_{\rho_b}  \, + \\
    & \biggl|  \frac{2}{9} \, (1 - \frac{\rho}{\rho_b}) \, g \, \frac{\rho_b}{v^2} \, r^2 \biggr|  \, \epsilon_v \, + \\
    & \biggl| \frac{4}{9}\, (1 - \frac{\rho}{\rho_b}) \, g \, \frac{\rho_b}{v} \, r \biggr|  \, \epsilon_r
\end{align*}

