\section{Procedimiento}\label{sec:resultados}

\subsection{Primer experimento: Determinaci�n del radio de la gota de agua}\label{subsec:primer}

La tabla~\ref{tab:radio} muestra 5 mediciones del volumen total para un n�mero de gotas determinado.
El radio $r_i$ de una gota se calcula a partir del volumen total y dividiendo por el n�mero de gotas.

\begin{table}[h!]
    \caption{Radio $r_i$ de las gotas de agua producidas por el cuentagotas.}
    \label{tab:radio}
    \begin{centering}
        \begin{tabular}{|P{60px}|P{68px}|P{64px}|}
            \hline
            n� de gotas & volumen (cm$^3$) & $r_i\,$(mm)                           \\
            \hline
            \csvreader[late after line= \\, /csv/separator=semicolon ]{./files/data/radio.csv}{}% use head of csv as column names
            {\csvcoli   & \csvcolii        & \csvcoliv    }% specify your columns here
            \hline
        \end{tabular}
    \end{centering}
\end{table}

El radio medio de una gota con el error debido a la dispersi�n observada es:
\begin{equation*}
    r = (1.91 \pm 0.02)\, \text{mm}
\end{equation*}

La dispersi�n del valor medio del radio $r$ se ha calculado como:
\begin{equation*}
    \sigma_r = \sqrt{\frac{\sum (r -r_i)^2}{n(n-1)}}
\end{equation*}

Con este dato, determinamos el volumen de una gota de agua con su error asociado mediante las expresiones:
\begin{equation*}
    V = \frac{4}{3}\pi r^3 \hspace{10px}; \hspace{10px}
    \epsilon_V = 4\pi r^2 \epsilon_r
\end{equation*}

El valor del volumen $V$ es:
\begin{equation*}
    V = (29 \pm 2)\, \text{mm}^3
\end{equation*}


\FloatBarrier

\subsection{Segundo experimento: Determinaci�n de la densidad del aceite}\label{subsec:segundo}

En este experimento, llenamos un vaso de pl�stico con $V = 10\,$cm$^3$ de aceite.
A continuaci�n, llenamos el otro vaso de pl�stico con un volumen $V_a$ de agua y a�adimos $n$ gotas hasta que se equilibra la balanza.
Utilizamos el valor del volumen de una gota de agua calculado en el apartado~\ref{subsec:primer} para determinar el volumen
total de agua $V_T$ a partir de $n$.

Por �ltimo, la relaci�n entre el volumen de aceite $V$ y el volumen de agua $V_T$ nos permite hallar la relaci�n entre
sus densidades $\rho / \rho_b$.

La tabla~\ref{tab:densidad} muestra las medidas y resultados.

\begin{table}[h!]
    \caption{Volumen de aceite frente a volumen de agua que equilibra la balanza y relaci�n de sus densidades.}
    \label{tab:densidad}
    \begin{centering}
        \begin{tabular}{|P{36px}|P{36px}|P{24px}|P{36px}|P{36px}|}
            \hline
            $V\,(\text{cm}^3)$ & $V_a\,(\text{cm}^3)$ & $n$        & $V_T\,(\text{cm}^3)$ & $(\rho/\rho_b)_i$                     \\
            \hline
            \csvreader[late after line= \\, /csv/separator=semicolon ]{./files/data/densidad.csv}{}% use head of csv as column names
            {\csvcoli          & \csvcolii            & \csvcoliii & \csvcoliv            & \csvcolvi  }% specify your columns here
            \hline
        \end{tabular}
    \end{centering}
\end{table}

Utilizando la misma expresi�n de la dispersi�n que en la secci�n~\ref{subsec:primer}
para determinar el error asociado al valor medio de la relaci�n entre densidades, obtenemos un
resultado de:
\begin{equation*}
    \frac{\rho}{\rho_b} = 0.986 \pm 0.006
\end{equation*}

Recordando que el valor de la densidad del agua en el Sistema Internacional de unidades es $\rho_b = 1000\, \text{kg/m}^3$,
obtenemos el siguiente resultado para la densidad del aceite:

\begin{equation*}
    \rho = (986 \pm 6)\, \text{kg/m$^3$}
\end{equation*}

\subsection{Tercer experimento: Determinaci�n de la velocidad de ca�da de la gota de agua}\label{subsec:tercero}

En este experimento, se dejan caer 50 gotas de agua en una botella de aceite y se mide el tiempo $t$
que tardan en recorrer la distancia entre dos marcas, $d = 10.1\,$cm.
La tabla~\ref{tab:velocidad} muestra las medidas del tiempo $t$ y la velocidad $v$ correspondiente.

\begin{table}[h!]
    \scriptsize
    \caption{Velocidad $v$ de una gota de agua}
    \label{tab:velocidad}
    \begin{centering}
        \begin{tabular}{|P{101px}|P{102px}|}
            \hline
            $t\,$(s)  & $v\,$ (cm/s)                           \\
            \hline
            \csvreader[late after line= \\, /csv/separator=semicolon ]{./files/data/velocidad.csv}{}% use head of csv as column names
            {\csvcoli & \csvcoliv }% specify your columns here
            \hline
        \end{tabular}
    \end{centering}
\end{table}

El valor medio de $v$ con su error calculado como en las secciones~\ref{subsec:primer} y~\ref{subsec:segundo} es:
\begin{equation*}
    v = -(0.71 \pm 0.02)\,\text{cm/s}
\end{equation*}

\subsection{C�lculo de la viscosidad del aceite}

Ya estamos en disposici�n de utilizar la ecuaci�n~\ref{eq:viscosity} para calcular el valor de la viscosidad $\eta$.

Sustituimos en esta ecuaci�n los valores medios de $r$, $\rho$ y $v$ calculados en las
secciones~\ref{subsec:primer},~\ref{subsec:segundo} y~\ref{subsec:tercero}.
A su vez, obtenemos una expresi�n del error asociado de la viscosidad $\epsilon_{\eta}$ como medida indirecta
a partir de las derivadas parciales de la ecuaci�n~\ref{eq:viscosity} y de los errores asociados a cada variable:
\begin{align*}
    \epsilon_{\eta} = & \biggl| \frac{2}{9} \, \frac{g\,r^2}{v} \biggr|  \, \epsilon_{\rho}  \, + \\
    & \biggl|  \frac{2}{9} \, (1 - \frac{\rho}{\rho_b}) \, g \, \frac{\rho_b}{v^2} \, r^2 \biggr|  \, \epsilon_v \, + \\
    & \biggl| \frac{4}{9}\, (1 - \frac{\rho}{\rho_b}) \, g \, \frac{\rho_b}{v} \, r \biggr|  \, \epsilon_r
\end{align*}

Con todo ello, el valor de la viscosidad obtenido es:
\begin{equation*}
    \eta = (0.015 \pm 0.009)\,\text{kg/m$\cdot$s}
\end{equation*}

