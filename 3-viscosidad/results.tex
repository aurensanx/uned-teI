\section{Procedimiento}\label{sec:resultados}

\subsection{Primer experimento: Determinaci�n del radio de la gota de agua}


\begin{table}[h!]
    \caption{Radio $r_i$ de las gotas de agua producidas por el cuentagotas}
    \label{tab:radio}
    \begin{centering}
        \begin{tabular}{|P{64px}|P{64px}|P{64px}|}
            \hline
            n� de gotas & volumen (cm$^3$) & $r_i\,$(mm)                           \\
            \hline
            \csvreader[late after line= \\, /csv/separator=semicolon ]{./files/data/radio.csv}{}% use head of csv as column names
            {\csvcoli   & \csvcolii        & \csvcoliv    }% specify your columns here
            \hline
        \end{tabular}
    \end{centering}
\end{table}


\FloatBarrier

\subsection{Segundo experimento: Determinaci�n de la densidad del aceite}

\begin{table}[h!]
    \caption{Volumen de aceite frente a volumen del agua que equilibra la balanza}
    \label{tab:densidad}
    \begin{centering}
        \begin{tabular}{|P{64px}|P{64px}|P{64px}|}
            \hline
            $V\,(\text{cm}^3)$ & $V_T\,(\text{cm}^3)$ & $(\rho/\rho_b)_i$                     \\
            \hline
            \csvreader[late after line= \\, /csv/separator=semicolon ]{./files/data/densidad.csv}{}% use head of csv as column names
            {\csvcoli          & \csvcolii            & \csvcoliv    }% specify your columns here
            \hline
        \end{tabular}
    \end{centering}
\end{table}

\subsection{Tercer experimento: Determinaci�n de la velocidad de ca�da de la gota de agua}

\subsection{C�lculo de la viscosidad del aceite}

\begin{align*}
    \epsilon_{\eta} = & \biggl| \frac{2}{9} \, \frac{g\,r^2}{v} \biggr|  \, \epsilon_{\rho_b}  \, + \\
    & \biggl|  \frac{2}{9} \, (1 - \frac{\rho}{\rho_b}) \, g \, \frac{\rho_b}{v^2} \, r^2 \biggr|  \, \epsilon_v \, + \\
    & \biggl| \frac{4}{9}\, (1 - \frac{\rho}{\rho_b}) \, g \, \frac{\rho_b}{v} \, r \biggr|  \, \epsilon_r
\end{align*}

