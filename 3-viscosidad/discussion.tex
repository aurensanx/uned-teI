\section{An�lisis y conclusiones}

El valor de la viscosidad $\eta$ obtenido est� bastante alejado del determinado en laboratorios profesionales ($\eta = 0.084\,\text{kg/m$\cdot$s}$).

En esta secci�n sugerimos una serie de ideas para intentar justificar el porqu� no hemos obtenido el mismo resultado.

Observamos que el valor obtenido de la densidad del aceite $\rho = (986 \pm 6)\, \text{kg/m$^3$}$ es mayor al esperado ($\rho = 920\, \text{kg/m$^3$}$).
Si utilizamos el valor de $\rho$ real en la ecuaci�n~\ref{eq:viscosity},
manteniendo el resto de par�metros constantes, obtenemos un nuevo valor de viscosidad $\eta = 0.09\,\text{kg/m$\cdot$s}$, que s� se aproxima al real.
Esto nos sugiere que el mayor error en el procedimiento puede originarse en el experimento 2 del c�lculo de la densidad del aceite.

Una posible causa es que hubiese restos de s�lidos en el vaso de pl�stico de aceite debido a experimentos previos.
Un mayor peso en el vaso de aceite que el que corresponder�a al volumen $V$ a�adido significar�a que se
necesita un mayor volumen de agua $V_a$ para equilibrar la balanza.
En ese caso, el valor calculado de la densidad de aceite $\rho$  ser�a mayor al real, como sucede en nuestras medidas.

Otra posible fuente de error es el experimento 3 de determinaci�n de la velocidad $v$ de una gota de agua.
No exist�a una abertura limpia en la botella de aceite para dejar caer las gotas de agua.
Si alguna gota en su ca�da golpease los bordes de las ranuras de la abertura, como as� suced�a, y su radio $r$ disminuyese porque la gota se dividiese,
esto afectar�a al resultado del experimento.

%Esto tambi�n parece tener un efecto, ya que
%si utilizamos un valor de $r = 1.60\,$mm, menor que el obtenido de $r=1.91\,$mm, en la ecuaci�n~\ref{eq:viscosity}, actualizando a su vez
%el valor de la densidad $\rho_b$ debido al nuevo radio $r$,
%la nueva viscosidad obtenida es $\eta = 0.05\,\text{kg/m$\cdot$s}$,
%un valor m�s ajustado al esperado.

Otra posible fuente de error es que el aceite utilizado no fuese aceite puro, sino que estuviese mezclado con suficiente agua de previos
experimentos como para influir en las mediciones.

Una �ltima causa puede ser debida a que, para romper la tensi�n superficial de aceite y que las gotas de agua cayesen, fue necesario
sacudir levemente la botella de agua en varias ocasiones.
Un l�quido en movimiento afectar�a tambi�n a la velocidad de ca�da de la gota de agua y, por tanto, al resultado del experimento.


