\section{Introducci�n}

El movimiento de ca�da en un l�quido de una esfera m�s densa que dicho l�quido viene dado por la ecuaci�n:
\begin{equation}
    \label{eq:newton}
    m_b\,\frac{d^2 z}{dt^2} = - m_b\, g + m\,g - 6\,\pi\,\eta\,r\,\frac{dz}{dt}
\end{equation}
donde los t�rminos del segundo miembro de la ecuaci�n son respectivamente: el peso de la esfera, el empuje debido al volumen de l�quido
desalojado y la fuerza de viscosidad que se opone al movimiento.

La soluci�n a esta ecuaci�n diferencial viene dado por las expresiones:
\begin{align}
    \label{eq:solution}
    z(t) &= z_0 + \beta \, (1 - e^{-\alpha t} -\alpha \,t) \\
    v(t) &= - \alpha \, \beta \, (1 - e^{-\alpha t})
\end{align}
con constantes:
\begin{equation*}
    \alpha = \frac{9}{2} \, \frac{\eta}{r^2\,\rho_b} \hspace{10px};  \hspace{10px}
    \beta = \biggl(  1 - \frac{\rho}{\rho_b} \biggr)\, \frac{g}{\alpha^2}
\end{equation*}

Si $\alpha$ es grande (viscosidad elevada o radio muy peque�o), el t�rmino exponencial de las ecuaciones de movimiento
disminuye muy r�pidamente y el movimiento se aproxima a un movimiento uniforme, con velocidad:
\begin{equation}
    \label{eq:velocity}
    v = - \alpha \, \beta = - \frac{2}{9} \, \biggl(1 - \frac{\rho}{\rho_b} \biggr) \, g \, \frac{\rho_b}{\eta} \, r^2
\end{equation}

A partir de la velocidad $v$ podemos determinar la viscosidad como:
\begin{equation}
    \label{eq:viscosity}
    \eta = - \frac{2}{9} \, \biggl(1 - \frac{\rho}{\rho_b} \biggr) \, g \, \frac{\rho_b}{v} \, r^2
\end{equation}



