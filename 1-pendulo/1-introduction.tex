\section{Introducci�n}

El p�ndulo simple es un sistema f�sico en el que se genera un movimiento oscilatorio al transformarse repetidamente energ�a potencial en cin�tica y viceversa, debido
a la acci�n de la fuerza gravitatoria sobre la masa $m$ del p�ndulo.

Para oscilaciones no demasiado grandes, el movimiento oscilatorio del p�ndulo es arm�nico simple, con periodo:
\begin{equation}
    \label{eq:1}
    T = 2 \pi \sqrt{\frac{l}{g}}
\end{equation}

Para �ngulos grandes de movimiento, la aproximaci�n $\sin \theta \approx \theta$ ya no es v�lida, y el movimiento del p�ndulo deja de ser arm�nico simple.

Los efectos inevitables de rozamiento, empuje hidrost�tico del aire y masa no despreciable del hilo del p�ndulo
se minimizan si tomamos una masa $m$ suficientemente grande para realizar las mediciones.


