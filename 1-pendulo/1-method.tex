\section{Dispositivo experimental}

El experimento comienza atando una masa $m$ a un extremo de un hilo inextensible (sedal de ca�a de pescar).
El otro extremo del hilo se ata al soporte.

Medimos a su vez el di�metro de la masa $m$, y la longitud del hilo m�s el radio de la masa ser� la longitud $l$ del movimiento arm�nico.

La precisi�n de la regla es $1\,$mm, por lo que el error absoluto de las medidas de longitud ser� $\pm\; 0,5\,$mm.

Con un transportador graduado, medimos el �ngulo $\theta$ respecto a la vertical (ver figura~\ref{fig:pendulum}) y soltamos la masa $m$.

\usetikzlibrary{calc,patterns,angles,quotes}

\begin{figure}[tbh]
    \begin{center}
        \begin{tikzpicture}[black!75]

            % Supporting structure
            \fill [pattern = north west lines] (0,0) rectangle ++(3,.2);
            \draw[thick] (0,0) -- ++(3,0);

            \draw[dashed,gray] (1.5,0)coordinate(o) -- ++(0,-3)coordinate(b);
            \draw (o) -- ++(290:3cm)coordinate(m) node[right=6pt] {$m$};
            \node[circle,draw,fill=grey!30] at (m);

            \pic [draw, ->, "$\theta$", angle eccentricity=1.5] {angle = b--o--m};

            \node[below=24pt, right=-2pt] at (o)   {$\theta$}

        \end{tikzpicture}
        \caption{Diagrama del p�ndulo simple}
        \label{fig:pendulum}
    \end{center}
\end{figure}

Contamos el tiempo que tardan en completarse $n$ oscilaciones, para minimizar el error en el c�lculo del periodo.

Adem�s, repetimos cada medici�n 3 veces para disminuir el error al accionar el cron�metro.

Hacemos lo mismo para 8 longitudes distintas, aproximadamente equiespaciadas, para dos �ngulos, 20\textdegree \;y 45\textdegree, y para dos masas.