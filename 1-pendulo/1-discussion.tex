\section{Cuestiones}\label{sec:discusion}

\subsection{Discusi�n de los resultados de $g$}

Las tablas~\ref{tab:1-t4-20},~\ref{tab:1-t4-45},~\ref{tab:2-t4-20} y~\ref{tab:2-t4-45} muestran los valores individuales de $g$ calculados para cada medida.

\begin{table}[h!]
    \caption{Masa 1 - $20 \text{\textdegree}$ - $g$.}
    \label{tab:1-t4-20}
    \begin{centering}
        \begin{tabular}{|P{56px}|P{67px}|P{67px}|}
            \hline
            $l$\,(mm)  & $T$\,(s)   & $g$\,(m/s)                             \\
            \hline
            \csvreader[late after line= \\, /csv/separator=semicolon ]{./files/data/1-t-20_es.csv}{}% use head of csv as column names
            {\csvcolii & \csvcolxii & \csvcolxiv}% specify your columns here
            \hline
        \end{tabular}
    \end{centering}
\end{table}

\begin{table}[h!]
    \caption{Masa 1 - $45 \text{\textdegree}$ - $g$.}
    \label{tab:1-t4-45}
    \begin{centering}
        \begin{tabular}{|P{56px}|P{67px}|P{67px}|}
            \hline
            $l$\,(mm)  & $T$\,(s)   & $g$\,(m/s)                             \\
            \hline
            \csvreader[late after line= \\, /csv/separator=semicolon ]{./files/data/1-t-45_es.csv}{}% use head of csv as column names
            {\csvcolii & \csvcolxii & \csvcolxiv}% specify your columns here
            \hline
        \end{tabular}
    \end{centering}
\end{table}


El error absoluto de $g$ se calcula con el m�todo de propagaci�n lineal de errores mediante la expresi�n:

\begin{equation}
    \epsilon_g = 4\pi^2 \biggl( \frac{\epsilon_l}{T^2} + \frac{2 l}{T^3}\, \epsilon_T \biggr)
\end{equation}

siendo el error absoluto de la longitud $\epsilon_l = 0.0005\,$m (se omite en las tablas),
y el error del periodo $\epsilon_T = 0.01\,$s para 20 oscilaciones y $\epsilon_T = 0.02\,$s para 10.

Se muestran m�s d�gitos de los usuales en el error para observar mejor la dependencia con la longitud $l$.


\begin{table}[h!]
    \caption{Masa 2 - $20 \text{\textdegree}$ - $g$.}
    \label{tab:2-t4-20}
    \begin{centering}
        \begin{tabular}{|P{56px}|P{67px}|P{67px}|}
            \hline
            $l$\,(mm)  & $T$\,(s)   & $g$\,(m/s)                             \\
            \hline
            \csvreader[late after line= \\, /csv/separator=semicolon ]{./files/data/2-t-20_es.csv}{}% use head of csv as column names
            {\csvcolii & \csvcolxii & \csvcolxiv}% specify your columns here
            \hline
        \end{tabular}
    \end{centering}
\end{table}

\begin{table}[h!]
    \caption{Masa 2 - $45 \text{\textdegree}$ - $g$.}
    \label{tab:2-t4-45}
    \begin{centering}
        \begin{tabular}{|P{56px}|P{67px}|P{67px}|}
            \hline
            $l$\,(mm)  & $T$\,(s)   & $g$\,(m/s)                             \\
            \hline
            \csvreader[late after line= \\, /csv/separator=semicolon ]{./files/data/2-t-45_es.csv}{}% use head of csv as column names
            {\csvcolii & \csvcolxii & \csvcolxiv}% specify your columns here
            \hline
        \end{tabular}
    \end{centering}
\end{table}

\FloatBarrier

Los valores de $g$ calculados a partir de la pendiente se recogen en las tablas~\ref{tab:g} y~\ref{tab:g2} ya mencionadas.

Observamos que los datos de $g$ para cada medici�n tienen errores absolutos apreciables, mayores que los calculados a partir de la pendiente.

Fij�ndonos en los valores calculados a partir de la pendiente,
para el movimiento con $\theta = 20\text{\textdegree}\,$, los valores calculados son muy cercanos al valor real de $g$, estando en todos los casos
dentro del margen de error o muy cerca de �l.

Sin embargo, para el �ngulo $\theta = 45\text{\textdegree}\,$, el mayor periodo del movimiento hace que se obtenga un valor $g$ menor del esperado.

\subsection{Dependencia de la incertidumbre de $g$ con $l$}

El error absoluto de $l$ es el mismo en todas las mediciones.
Para valores altos de $l$, el error relativo cometido en la medici�n es menor.
Esto se traduce, seg�n se puede observar en las tablas~\ref{tab:1-t4-20},~\ref{tab:1-t4-45},~\ref{tab:2-t4-20} y~\ref{tab:2-t4-45},
en un descenso de la incertidumbre de $g$ a medida que aumenta $l$.