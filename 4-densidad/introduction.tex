\section{Introducci�n}

El principio de Arqu�medes afirma que un cuerpo sumergido en un l�quido experimenta una fuerza vertical de empuje igual al peso del volumen del l�quido desalojado.

Si se sumerge un s�lido de volumen $V$ dentro de un l�quido, su peso aparente ser�:
\begin{equation}
    \label{eq:p_ap}
    P_{ap} = P - E = (\rho_s - \rho_l) \, V \, g
\end{equation}

Para determinar la densidad del s�lido, se puede pesar el s�lido en el aire y posteriormente sumergido en el l�quido, y utilizar la relaci�n:
\begin{equation}
    \label{eq:d_s}
    \frac{\rho_s}{\rho_l} = \frac{P}{E} = \frac{P}{P - P_{ap}}
\end{equation}

Si se vuelve a pesar el s�lido sumergido en otro l�quido, obtenemos un peso aparente $P'_{ap}$ y podemos determinar la densidad del nuevo l�quido
sin conocer la densidad del s�lido:
\begin{equation}
    \label{eq:d_l}
    \frac{\rho'_l}{\rho_l} = \frac{P - P'_{ap}}{P - P_{ap}}
\end{equation}

Adicionalmente, podemos medir con mayor precisi�n la densidad de un s�lido utilizando un suspensor de masas y a�adiendo sucesivamente pesas de masa $m$.
En ese caso, el peso del sistema en el aire y sumergido en agua ser�:
\begin{equation}
    \label{eq:masas_aire}
    P = n \, \rho_s \, V_s \, g + \rho_{susp} \, V_{susp} \, g
\end{equation}
\begin{equation}
    \label{eq:masas_agua}
    P_{ap} = n \, (\rho_s -\rho_l) \, V_s \, g + (\rho_{susp} - \rho_l) \, V_{susp} \, g
\end{equation}

Se pueden obtener los valores buscados de las densidades $\rho_s$ y $\rho_{susp}$ representando las rectas~\ref{eq:masas_aire} y~\ref{eq:masas_agua}
frente al n�mero de masas $n$.