\section{Cuestiones}\label{sec:discusion}

\subsection{}

Si consideramos el efecto del aire en el recipiente, el valor $P_{aire}$ que marca el dinam�metro es menor que
lo que marcar�a en condiciones ideales en el vac�o.

Si hacemos una estimaci�n, suponiendo que el cilindro hueco tiene una cavidad de $10\,$cm de alto con una base circular de
$2\,$cm de radio, y sabiendo que $\rho_{aire} = 1.204\n$kg/m$^3$, obtenemos que el dinam�metro marca $0.006\,$N menos de
lo que har�a en el vac�o.

Esta diferencia es del mismo orden de magnitud que el margen de error del dinam�metro, as� que es complicado percibir sus
efectos con seguridad.

Sin embargo, en las tablas~\ref{tab:c-1} y~\ref{tab:c-2}, se puede apreciar que $P_{vac}$, que en este caso ser�a $P_{aire}$, es menor
que $P_{ap} + P_{agua}$.

El efecto del aire puede contribuir a esta discrepancia, adem�s del efecto ya comentado de la precisi�n del dinam�metro.

\subsection{}

