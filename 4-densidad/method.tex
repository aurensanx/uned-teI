\section{Dispositivo experimental}

El dispositivo utilizado se muestra en la figura~\ref{fig:dispositivo}.

\colorlet{mydarkblue}{blue!50!black}
\colorlet{watercol}{blue!80!cyan!10!white}
\colorlet{darkwatercol}{blue!80!cyan!20!white}
\tikzstyle{water}=[draw=mydarkblue,top color=watercol!90,bottom color=watercol!90!black,shading angle=5]
\tikzstyle{vertical water}=[water,
top color=watercol!90!black!90,bottom color=watercol!90!black!90,middle color=watercol!80,shading angle=90]

\begin{figure}[tbh]
    \begin{center}
        \begin{tikzpicture}[black!75]

            % Supporting structure
            \fill [pattern = north west lines] (0,0)coordinate(origin) rectangle ++(3,.2);
            \draw[thick] (0,0) -- ++(3,0);

            %argola
            \draw[line ] (1.5,0) -- ++(0,-.2)coordinate(dynamo);

            %espelho do dinam�metro
            \draw[line] (1.6,-.3) arc (0:180:.1) -- ++(0,-1) arc (180:360:.1) -- cycle;

            %gancho
            \draw[line] (1.5,-1.4) -- ++(0,-.1) arc (90:360:.1);

            \draw[line ] (1.5,-1.7) -- ++(0,-.2);

            \coordinate (hbl) at (1.2,-2.7);
            \coordinate (hbr) at (1.8,-2.7)
            \coordinate (htl) at (1.2,-1.9);
            \coordinate (htr) at (1.8,-1.9);
            %cylinder
            \draw [](hbl) -- ++(0,0.8);
            \draw [](hbr) -- ++(0,0.8);
            \draw [](hbl) arc (180:360:0.3 and 0.1);          % <--
            \draw[dash pattern=on 2pt off 1pt] (hbr) arc (-0.3:180:0.3 and 0.1);  % <--
            \draw [](htl) arc (180:360:0.3 and 0.1);          % <--
            \draw [](htr) arc (-0.3:180:0.3 and 0.1);         % <--
%hollow

            \coordinate (fbl) at (1.3,-2.7);
            \coordinate (fbr) at (1.7,-2.7)
            \coordinate (ftl) at (1.3,-1.9);
            \coordinate (ftr) at (1.7,-1.9);

            \draw [](fbl) -- ++(0,0.8);
            \draw [](fbr) -- ++(0,0.8);
            \draw [](fbl) arc (180:360:0.2 and 0.06666);
            \draw[dash pattern=on 2pt off 1pt] (fbr) arc (-1.65:180:0.2 and 0.06666);
            \draw [](ftl) arc (180:360:0.2 and 0.06666);
            \draw [](ftr) arc (-0.2:180:0.2 and 0.06666);

            \draw[line] (1.5,-2.7) -- ++(0,-.3);

            \coordinate (fbl) at (1.3,-3.8);
            \coordinate (fbr) at (1.7,-3.8)
            \coordinate (ftl) at (1.3,-3);
            \coordinate (ftr) at (1.7,-3);

            \draw [](fbl) -- ++(0,0.8);
            \draw [](fbr) -- ++(0,0.8);
            \draw [](fbl) arc (180:360:0.2 and 0.06666);
            \draw[dash pattern=on 2pt off 1pt] (fbr) arc (-1.65:180:0.2 and 0.06666);
            \draw [](ftl) arc (180:360:0.2 and 0.06666);
            \draw [](ftr) arc (-0.2:180:0.2 and 0.06666);


            \def\Rx{0.45}      % tank horizontal radius
            \def\Ry{0.15}      % tank vertical radius
            \def\L{1.5-\Rx}      % tank vertical radius
            \def\R{1.5+\Rx}      % tank vertical radius
            \def\H{1.2}       % height tank
            \def\h{0.84*\H}   % height water
            \def\y{-5.4}   % vertical position piece


            % WATER
            \draw[vertical water]
            (\L,\y+\h) -- (\L,\y) arc (180:360:{\Rx} and {\Ry}) -- (\R,\y+\h);
            \draw[water, very thin]
            (\L+\Rx,\y+\h) ellipse ({\Rx} and {\Ry});

            % CONTAINER
            \draw[]
            (\L,\y+\H) -- (\L,\y) arc (180:360:{\Rx} and {\Ry}) -- (\R,\y+\H);
            \draw[]
            (\L+\Rx,\y+\H) ellipse ({\Rx} and {\Ry});


        \end{tikzpicture}
        \caption{Diagrama del dispositivo utilizado}
        \label{fig:dispositivo}
    \end{center}
\end{figure}

En primer lugar, se cuelga un dinam�metro del soporte.

A continuaci�n, se cuelgan dos cilindros, el primero hueco y el segundo macizo, de manera que el cilindro macizo encaja perfectamente en la cavidad del cilindro hueco.
En esta pr�ctica utilizaremos dos cilindros macizos de distinta masa.

Finalmente, se rellena un recipiente del l�quido del que se quieran tomar las medidas; en nuestro caso: agua, glicerina y vino.