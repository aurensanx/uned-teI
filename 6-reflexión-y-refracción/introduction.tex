\section{Introducci�n}

En esta pr�ctica estudiaremos la reflexi�n y refracci�n de la luz en la superficie de separaci�n de dos medios diel�ctricos transparentes.

Para ello, es necesario conocer previamente algunos t�rminos:

\begin{itemize}
    \item[$\bullet$] El \textit{punto de incidencia} es el punto en el que incide el rayo sobre la superficie de separaci�n de los dos medios, que es el mismo
    del que salen los rayos reflejado y refractado.
    \item[$\bullet$] Los �ngulos de incidencia $\theta_i$, de reflexi�n $\theta_r$ y de refracci�n (o transmisi�n) $\theta_t$ son los que forman,
    respectivamente, el rayo incidente, el rayo reflejado y el rayo refractado con la normal en el punto de incidencia.
\end{itemize}

Con esto, las leyes de reflexi�n y refracci�n se resumen en:

\begin{enumerate}
    \item El rayo incidente, el rayo reflejado, el rayo refractado y la normal est�n en un mismo plano que se denomina \textit{plano de incidencia}.
    \item Los �ngulos de incidencia y de reflexi�n son iguales (\textit{ley de la reflexi�n}):
    \begin{equation}
        \theta_i = \theta_r
    \end{equation}
    \item El seno del �ngulo de refracci�n es proporcional al seno del �ngulo de incidencia (\textit{ley de la refracci�n}):
    \begin{equation}
        \sen{\theta_r} = \frac{1}{n}\, \sen{\theta_i}
    \end{equation}
    donde $n$ es el �ndice de refracci�n del medio hacia el que se transmite la luz respecto al medio de donde proviene.
    \item Las leyes de la reflexi�n y refracci�n son reversibles.
    Como consecuencia, cuando la luz pasa de un medio de �ndice de refracci�n absoluto mayor a otro de �ndice menor,
    se produce el fen�meno de la \textit{reflexi�n total} a partir de un \textit{�ngulo l�mite}.
\end{enumerate}



