\section{Dispositivo experimental}

Para llevar a cabo la pr�ctica, utilizaremos el siguiente material:

\begin{itemize}
    \item[$\bullet$] Un semic�rculo o rodaja de vidrio transparente.
    \item[$\bullet$] Papel marcado con una escala graduada para medir �ngulos.
    \item[$\bullet$] L�ser.
\end{itemize}

Tomaremos medidas con dos configuraciones distintas del dispositivo.

En la primera, el rayo incidente se propaga por el aire y el rayo refractado por el vidrio, de modo que se acerca a la normal.
Esta configuraci�n se muestra en la figura~\ref{fig:dispositivo1}.

\usetikzlibrary{arrows,shapes,positioning}
\usetikzlibrary{decorations.markings}
\tikzstyle arrowstyle=[scale=1]
\tikzstyle directed=[postaction={decorate,decoration={markings,
mark=at position .65 with {\arrow[arrowstyle]{stealth}}}}]
\tikzstyle reverse directed=[postaction={decorate,decoration={markings,
mark=at position .65 with {\arrowreversed[arrowstyle]{stealth};}}}]

\begin{figure}[tbh!]
    \begin{center}
        \begin{tikzpicture}

            % define coordinates
            \coordinate (O) at (0,0);
            \coordinate (A) at (0,4);
            \coordinate (B) at (0,-3);
            \coordinate (TL) at (-4,4);
            \coordinate (TR) at (4,4);
            \coordinate (BL) at (-4,-3);
            \coordinate (BR) at (4,-3);

            \draw[thick] (-3,0) -- (3,0) arc(0:180:3) --cycle;
            \begin{scope}
                \clip (-3,0) rectangle (3,3);
                \fill[blue!60!,opacity=.3] (0,0) circle(3);
            \end{scope}

            % media
            \fill[blue!25!,opacity=.3] (TL) rectangle (BR);

            \node[right] at (1,0.5) {Vidrio};
            \node[left] at (-2.5,2.5) {Aire};

            % axis
            \draw[dash pattern=on5pt off3pt] (A) -- (B);

            % rays
            \draw[red,ultra thick,reverse directed] (O) -- (130:-3.9);
            \draw[blue,directed,ultra thick] (O) -- (-70:-4.24);
            \draw[green,directed,ultra thick] (O) -- (-130:3.9);

            % angles
            \draw (0,1) arc (90:110:1);
            \draw (0,-1) arc (270:310:1);
            \draw (0,-1) arc (270:230:1);
            \node[] at (290:1.4)  {$\theta_{i}$};
            \node[] at (250:1.4)  {$\theta_{r}$};
            \node[] at (100:1.4)  {$\theta_{t}$};
        \end{tikzpicture}
        \caption{Configuraci�n del dispositivo con el rayo incidente viajando por el aire.}
        \label{fig:dispositivo1}
    \end{center}
\end{figure}

\FloatBarrier

La segunda configuraci�n invierte la rodaja de vidrio, y la utilizaremos para calcular el �ngulo l�mite $\theta_l$ a partir del cual se
produce la reflexi�n total.
Esta configuraci�n se muestra en la figura~\ref{fig:dispositivo2}.


\begin{figure}[tbh!]
    \begin{center}
        \begin{tikzpicture}

            % define coordinates
            \coordinate (O) at (0,0);
            \coordinate (A) at (0,4);
            \coordinate (B) at (0,-2);
            \coordinate (TL) at (-4,4);
            \coordinate (TR) at (4,4);
            \coordinate (BL) at (-4,-2);
            \coordinate (BR) at (4,-2);

            \draw[thick] (-3,0) -- (3,0) arc(0:180:3) --cycle;
            \begin{scope}
                \clip (-3,0) rectangle (3,3);
                \fill[blue!60!,opacity=.3] (0,0) circle(3);
            \end{scope}

            % media
            \fill[blue!25!,opacity=.3] (TL) rectangle (BR);

            \node[right] at (1,0.5) {Vidrio};
            \node[left] at (-2.5,2.5) {Aire};

            % axis
            \draw[dash pattern=on5pt off3pt] (A) -- (B);

            % rays
            \draw[red,ultra thick,reverse directed] (O) -- (-130:-5.2);
            \draw[blue,directed,ultra thick] (O) -- (20:-4.24);
            \draw[green,directed,ultra thick] (O) -- (130:5.2);

            % angles
            \draw (0,1) arc (90:50:1);
            \draw (0,1) arc (90:130:1);
            \draw (0,-1) arc (270:200:1);
            \node[] at (70:1.4)  {$\theta_{i}$};
            \node[] at (235:1.4)  {$\theta_{t}$};
            \node[] at (110:1.4)  {$\theta_{r}$};
        \end{tikzpicture}
        \caption{Configuraci�n del dispositivo con el rayo incidente viajando por el vidrio.}
        \label{fig:dispositivo2}
    \end{center}
\end{figure}

