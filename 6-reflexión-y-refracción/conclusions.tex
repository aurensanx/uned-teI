\section{An�lisis y conclusiones}

\subsection{Ley de la reflexi�n}

La pendiente de la recta de la figura~\ref{fig:reflexion} es igual a la unidad, con una indeterminaci�n del $0.12\%$.

Como se ha comentado en la secci�n~\ref{subsubsec:ley-de-la-reflexion}, este resultado corrobora la validez de la ley de la reflexi�n.

\subsection{Ley de la refracci�n}\label{subsec:ley-de-la-refraccion}

La pendiente de la recta de la figura~\ref{fig:refraccion} es igual a $0.676 \pm 0.004$.

De acuerdo a la ecuaci�n~\ref{eq:refraccion}, se deduce el siguiente valor para el �ndice de refracci�n del vidrio:
\begin{equation*}
    n = 1.479 \pm 0.009
\end{equation*}

No sabemos exactamente de qu� tipo de vidrio est� fabricada la rodaja de nuestro dispositivo experimental, pero este
valor se encuentra dentro del rango de valores t�picos encontrados en la referencia~\cite{glass}.

\subsection{Aproximaci�n paraxial o de Gauss}

La pendiente de la recta de la figura~\ref{fig:aproximacion} es igual a $0.639 \pm 0.010$.

Por tanto, si utilizamos la aproximaci�n paraxial o de Gauss, el �ndice de refracci�n que obtenemos para el vidrio es:
\begin{equation*}
    n = 1.56 \pm 0.02
\end{equation*}

Este resultado es ligeramente mayor al encontrado en el apartado~\ref{subsec:ley-de-la-refraccion}.

Si realizamos ahora el ajuste lineal para �ngulos de incidencia $\theta_i$ menores que $30\text{\textdegree}$, el nuevo �ndice
de refracci�n del vidrio es:
\begin{equation*}
    n = 1.54 \pm 0.07
\end{equation*}

Este valor se acerca m�s al esperado.
Este hecho se explica porque estamos trabajando con �ngulos m�s peque�os, y la validez de la aproximaci�n paraxial o de Gauss aumenta.
Notamos tambi�n que aumenta el error de la medida, debido a disponer de menos datos sobre los que realizar el ajuste lineal.

\subsection{Reflexi�n total y �ngulo l�mite}

Como se ha explicado en la secci�n~\ref{subsec:tercero}, el �ngulo l�mite $\theta_l$ se produce cuando el �ngulo de refracci�n $\theta_t$ tiende a 90\textdegree.

De la ecuaci�n~\ref{eq:refraccion}, obtenemos un valor del �ndice de refracci�n para el vidrio de:
\begin{equation*}
    n = \frac{1}{\sen{\theta_l}} = 1.486 \pm 0.014
\end{equation*}

Este valor se encuentra dentro del rango obtenido en la secci�n~\ref{subsec:ley-de-la-refraccion} y tambi�n dentro del rango descrito en la referencia~\cite{glass}.

En resumen, los resultados mostrados corroboran la validez del experimento para estudiar las leyes elementales de la �ptica.
