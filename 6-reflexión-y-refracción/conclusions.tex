\section{An�lisis y conclusiones}

\subsection{Ley de la reflexi�n}

La pendiente de la recta de la figura~\ref{fig:reflexion} es igual a la unidad, con una indeterminaci�n del $0.12\%$.

\subsection{Ley de la refracci�n}

La pendiente de la recta de la figura~\ref{fig:refraccion} es igual a $0.676 \pm 0.004$.
De aqu� se deduce el siguiente valor para el �ndice de refracci�n del vidrio:
\begin{equation*}
    n = 1.479 \pm 0.009
\end{equation*}

\subsection{Aproximaci�n paraxial o de Gauss}

La pendiente de la recta de la figura~\ref{fig:aproximacion} es igual a $0.639 \pm 0.010$.
Esto corresponde a un �ndice de refracci�n para el vidrio de:
\begin{equation*}
    n = 1.56 \pm 0.02
\end{equation*}

\subsection{Reflexi�n total y �ngulo l�mite}

\begin{equation*}
    n = \frac{1}{\sen{\theta_t}} = 1.486 \pm 0.014
\end{equation*}
