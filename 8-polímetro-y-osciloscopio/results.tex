\section{Resultados}\label{sec:resultados}

\subsection{Medidas con el pol�metro}\label{subsec:medidas-con-el-polimetro}

\subsubsection{Medida de resistencias}\label{subsubsec:medidas-de-resistencias}

La primera medida a realizar con el pol�metro consiste en obtener los valores experimentales de varias resistencias suministradas.

Primero, determinamos el valor nominal $R_{nom}$ a partir del c�digo de colores.
A continuaci�n, seleccionando el modo para medir resistencias del pol�metro, se obtienen los valores experimentales $R_{exp}$.

La tabla~\ref{tab:resistencias} recoge los valores obtenidos.


\newcommand\stripe[2]{\draw[fill=#2,#2] ++(#1,0) +(-0.05,-0.11) rectangle +(0.05,0.11);}

\newcommand\resistor[4]{
    \begin{tikzpicture}
    \stripe{0}{#1}
    \stripe{0.2}{#2}
    \stripe{0.4}{#3}
    \stripe{0.8}{#4}
    \end{tikzpicture}
}

\definecolor{gold}{RGB}{255,215,0}

\begin{table}[h!]
    \center
    \caption{Valores nominales $R_{nom}$ y experimentales $R_{exp}$ de las resistencias.}
    \label{tab:resistencias}
    \begin{centering}
        \begin{tabular}{|P{40px}|P{60px}|P{50px}|}
            \hline
            & $R_{nom}$                  & $R_{exp}$        \\
            \hline
            \resistor{yellow}{violet}{brown}{gold}  & $470\,$\Omega$\,\pm\, 5\%$ & $0.469\,$k\Omega \\
            \resistor{red}{red}{black}{gold}        & $22\,$\Omega$\,\pm\, 5\%$  & $22.5\,$\Omega   \\
            \resistor{orange}{orange}{orange}{gold} & $33\,$k\Omega$\,\pm\, 5\%$ & $33.06\,$k\Omega \\
            \resistor{green}{brown}{brown}{red}     & $510\,$\Omega$\,\pm\, 2\%$ & $0.510\,$k\Omega \\
            \resistor{brown}{black}{brown}{gold}    & $100\,$\Omega$\,\pm\, 5\%$ & $98.9\,$\Omega   \\
            \hline
        \end{tabular}
    \end{centering}
\end{table}

Observamos que los valores experimentales $R_{exp}$ se encuentran dentro de la tolerancia de los valores nominales $R_{nom}$.

\subsubsection{Medida de voltaje en c.c.}\label{subsubsec:medida-de-voltaje-en-c.c.}

Tomamos ahora medidas de voltajes en corriente continua suministrados por la fuente de alimentaci�n.

Para ello, conectamos el pol�metro a la salida de la fuente de alimentaci�n en el modo de medici�n adecuado.

Utilizamos 3 escalas diferentes: 2\,V, 20\,V y 200\,V.
Si se sobrepasa el m�ximo que una escala permite medir, lo indicamos con un guion.

Las medidas tomadas se recogen en la tabla~\ref{tab:voltaje-cc}.

\begin{table}[h!]
    \center
    \caption{Tensi�n aplicada y medida en el pol�metro con distintas escalas.}
    \label{tab:voltaje-cc}
    \begin{centering}
        \begin{tabular}{|P{56px}|P{40px}|P{40px}|P{42px}|}
            \hline
            $\Delta V$ aplicada & $\Delta V$ 2V & $\Delta V$ 20V & $\Delta V$ 200V \\
            \hline
            $1.0\,$V            & $0.983\,$V    & $0.98\,$V      & $1.0\,$V        \\
            $2.0\,$V            & $1.974\,$V    & $1.97\,$V      & $1.9\,$V        \\
            $5.0\,$V            & -             & $5.00\,$V      & $5.0\,$V        \\
            $10.0\,$V           & -             & $10.04\,$V     & $10.0\,$V       \\
            $15.0\,$V           & -             & $15.09\,$V     & $15.0\,$V       \\
            $20.0\,$V           & -             & -              & $20.0\,$V       \\
            \hline
        \end{tabular}
    \end{centering}
\end{table}

Como podemos ver, la precisi�n de cada escala en el pol�metro es, en orden, $0.001\,$V, $0.01\,$V y $0.1\,$V.
La precisi�n de la fuente de alimentaci�n es $0.1\,$V.

\subsubsection{Medida de corrientes en c.c.}

En esta secci�n, utilizamos el pol�metro para tomar medidas de corriente continua.

Para ello, conectamos el pol�metro en serie a las resistencias medidas en el apartado~\ref{subsubsec:medidas-de-resistencias} y a la fuente de alimentaci�n,
como se indica en la figura~\ref{fig:circuito}.

\begin{figure}[tbh]
    \begin{center}
        \begin{circuitikz}
            \draw
            (0,0) to[battery] (0,2)
            to[R=$R$] (4,2) -- (4,0)
            (0,0) to[ammeter] (4,0)
        \end{circuitikz}
        \caption{Diagrama del circuito para medir corriente continua.}
        \label{fig:circuito}
    \end{center}
\end{figure}

Los valores de corriente te�ricos $i_{teor}$ y experimentales $i_{exp}$ se recogen en la tabla~\ref{tab:corriente-cc}.

\begin{table}[h!]
    \center
    \caption{Corriente te�rica $i_{teor}$ y experimental $i_{exp}$ con distintas resistencias $R$.}
    \label{tab:corriente-cc}
    \begin{centering}
        \begin{tabular}{|P{60px}|P{60px}|P{60px}|}
            \hline
            \multicolumn{3}{|c|}{$V = 5\,$V} \\
            \hline
            $R$              & $i_{teor} = V / R$ & $i_{exp}$     \\
            \hline
            $0.469\,$k\Omega & $10.66\,$mA        & $10.43\,$mA   \\
            $22.5\,$\Omega   & $222.2\,$mA        & $171.1\,$mA   \\
            $33.06\,$k\Omega & $151.2\,\mu$A      & $148.6\,\mu$A \\
            $0.510\,$k\Omega & $9.80\,$mA         & $9.53\,$mA    \\
            $98.9\,$\Omega   & $50.6\,$mA         & $46.6\,$mA    \\
            \hline
        \end{tabular}
    \end{centering}
\end{table}

\subsubsection{Medida de voltajes en c.a}

\subsection{Medidas con el osciloscopio}\label{subsec:medidas-con-el-osciloscopio}

\subsubsection{Medida de voltajes}
