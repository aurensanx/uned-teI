\section{Resultados}\label{sec:resultados}

\subsection{Medidas con el pol�metro}\label{subsec:medidas-con-el-polimetro}

\subsubsection{Medida de resistencias}\label{subsubsec:medidas-de-resistencias}

La primera medida a realizar con el pol�metro consiste en obtener los valores experimentales de varias resistencias suministradas.

Primero, determinamos el valor nominal $R_{nom}$ a partir del c�digo de colores.
A continuaci�n, seleccionando el modo para medir resistencias del pol�metro, se obtienen los valores experimentales $R_{exp}$.

La tabla~\ref{tab:resistencias} recoge los valores obtenidos.


\newcommand\stripe[2]{\draw[fill=#2,#2] ++(#1,0) +(-0.05,-0.11) rectangle +(0.05,0.11);}

\newcommand\resistor[4]{
    \begin{tikzpicture}
    \stripe{0}{#1}
    \stripe{0.2}{#2}
    \stripe{0.4}{#3}
    \stripe{0.8}{#4}
    \end{tikzpicture}
}

\definecolor{gold}{RGB}{255,215,0}

\begin{table}[h!]
    \center
    \caption{Valores nominales $R_{nom}$ y experimentales $R_{exp}$ de las resistencias.}
    \label{tab:resistencias}
    \begin{centering}
        \begin{tabular}{|P{40px}|P{60px}|P{50px}|}
            \hline
            & $R_{nom}$                  & $R_{exp}$        \\
            \hline
            \resistor{yellow}{violet}{brown}{gold}  & $470\,$\Omega$\,\pm\, 5\%$ & $0.469\,$k\Omega \\
            \resistor{red}{red}{black}{gold}        & $22\,$\Omega$\,\pm\, 5\%$  & $22.5\,$\Omega   \\
            \resistor{orange}{orange}{orange}{gold} & $33\,$k\Omega$\,\pm\, 5\%$ & $33.06\,$k\Omega \\
            \resistor{green}{brown}{brown}{red}     & $510\,$\Omega$\,\pm\, 2\%$ & $0.510\,$k\Omega \\
            \resistor{brown}{black}{brown}{gold}    & $100\,$\Omega$\,\pm\, 5\%$ & $98.9\,$\Omega   \\
            \hline
        \end{tabular}
    \end{centering}
\end{table}

Observamos que los valores experimentales $R_{exp}$ se encuentran dentro de la tolerancia de los valores nominales $R_{nom}$.

\subsubsection{Medida de voltaje en c.c.}\label{subsubsec:medida-de-voltaje-en-c.c.}

\subsubsection{Medida de corrientes en c.c.}

\subsubsection{Medida de voltajes en c.a}

\subsection{Medidas con el osciloscopio}\label{subsec:medidas-con-el-osciloscopio}

\subsubsection{Medida de voltajes}
