\section{Resultados}\label{sec:resultados}

La tabla~\ref{tab:R} recoge las medidas de $\Delta V$ para distintos valores de $I$ y 3 longitudes de cable diferentes:
$1\,$m, $1.5\,$m y $2\,$m.

Las mediciones se repiten para 3 tipos de cable distintos, con secciones de $1.5\,$mm$^2$, $0.5\,$mm$^2$ y $0.32\,$mm$^2$.

\begin{table}[h!]
    \scriptsize
    \caption{Tabla de datos. Intensidad de la corriente $I$ y diferencia de potencial $\Delta V$, para cables con
    longitudes $L$ y secciones $S$.}
    \label{tab:R}
    \begin{centering}
        \begin{tabular}{|P{13px}|P{18px}|P{26px}|P{18px}|P{26px}|P{18px}|P{26px}|}
            \hline
            \multicolumn{1}{|c}{} & \multicolumn{2}{|c}{$S = 1.5\,$mm$^2$} & \multicolumn{2}{|c}{$S = 0.5\,$mm$^2$} & \multicolumn{2}{|c|}{$S = 0.32\,$mm$^2$} \\
%            $L$       & $I$       & $\Delta V$       & $I$       & $\Delta V$       & $I$       & $\Delta V$                             \\
            \hline
            $L$(m)         & $I$(mA) & $\Delta V$(mV) & $I$(mA) & $\Delta V$(mV) & $I$(mA) & $\Delta V$(mV) \\
            \hline
            \multirow{1}   & 50      & 1,4            & 50      & 2,3            & 50      & 2,9            \\
            & 100     & 2,5            & 100     & 4,7            & 100     & 6,1            \\
            & 150     & 3,8            & 150     & 7,2            & 150     & 9,2            \\
            & 200     & 5,2            & 200     & 9,7            & 200     & 12,6           \\
            & 250     & 6,8            & 250     & 12,1           & 250     & 15,7           \\
            & 300     & 8,0            & 300     & 14,5           & 300     & 18,6           \\
            \hline
            \multirow{1.5} & 50      & 1,6            & 50      & 5,6            & 50      & 5,3            \\
            & 100     & 3,5            & 100     & 11,3           & 100     & 11,1           \\
            & 150     & 5,1            & 150     & 17,1           & 150     & 19,3           \\
            & 200     & 6,7            & 200     & 22,5           & 200     & 23,2           \\
            & 250     & 8,6            & 250     & 28,0           & 250     & 29,6           \\
            & 300     & 10,3           & 300     & 33,4           & 300     & 35,4           \\
            \hline
            \multirow{2}   & 50      & 1,7            & 50      & 6,9            & 50      & 6,2            \\
            & 100     & 3,5            & 100     & 13,4           & 100     & 12,4           \\
            & 150     & 5,3            & 150     & 20,1           & 150     & 19,0           \\
            & 200     & 7,2            & 200     & 27,4           & 200     & 25,2           \\
            & 250     & 9,0            & 250     & 33,7           & 250     & 31,2           \\
            & 300     & 10,9           & 300     & 40,1           & 300     & 38,3           \\
            \hline
        \end{tabular}
    \end{centering}
\end{table}


La precisi�n del amper�metro es de $10\,$mA y la del volt�metro de $0.1\,$mV, por tanto, los errores absolutos de las medidas de
intensidad y diferencia de potencial
son $\pm 5\,$mA y $\pm 0.05\,$mV respectivamente.

A la hora de tomar las medidas, se intent� utilizar una escala m�s peque�a en el amper�metro,
pero no se consigui� que por el circuito llegase a circular corriente.

