\section{Introducci�n}

La ley de Ohm es una ley emp�rica que establece que la diferencia de potencial $\Delta V$
que aplicamos entre los extremos de un conductor determinado es directamente proporcional a la intensidad de la corriente
$I$ que circula por el citado conductor.

Esta relaci�n se recoge en la ecuaci�n~\ref{eq:ohm}.
\begin{equation}
    \label{eq:ohm}
    \frac{\Delta V}{I} = R
\end{equation}

La resistencia $R$ de un conductor depende en primer lugar de la resistividad $\rho$, que es una propiedad intr�nseca del
material del que est� fabricado.

Adem�s, tambi�n depende de su geometr�a.
Para un cable conductor, que se asemeja a un cilindro de longitud $L$ y secci�n $S$, la relaci�n entre estos 3
factores la da la ecuaci�n~\ref{eq:resistividad}.
\begin{equation}
    \label{eq:resistividad}
    R = \rho\,\frac{L}{S}
\end{equation}