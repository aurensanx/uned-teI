\section{Dispositivo experimental}

El circuito que utilizamos para medir la resistencia del cable conductor puede verse en la figura~\ref{fig:circuito}.

\begin{figure}[tbh]
    \begin{center}
        \begin{circuitikz}
            \draw
            (0,0) to[battery] (0,2)
            to[R=$R$] (2,2)
            (4,2) -- (4.5,2) -- (4.5,0)
            (2,2) -- (2,1)
            to[voltmeter]  (4,1) -- (4,2)
            (0,0) to[ammeter]  (4.5,0)
            \draw[black, pattern={Lines[angle=45, line width=1pt]}, pattern color=orange] (2,2-0.075) rectangle ++(2,0.15)
        \end{circuitikz}
        \caption{Diagrama del circuito.}
        \label{fig:circuito}
    \end{center}
\end{figure}

En primer lugar, tenemos una fuente de energ�a, que suministra la corriente $I$ al circuito.

El objeto a rayas naranja de la figura~\ref{fig:circuito} representa el conductor de longitud $L$ y secci�n $S$ del que queremos
tomar las medidas.

En paralelo al conductor, conectamos un volt�metro para poder medir la diferencia de potencial $\Delta V$ entre sus extremos.

Adem�s, se conecta un amper�metro en serie a este conjunto para poder medir la corriente $I$ que circula por el circuito.

Por �ltimo, como la resistencia del cable conductor es muy peque�a, se a�ade una resistencia $R$ en serie para que la intensidad $I$ suministrada
por la fuente de energ�a no aumente peligrosamente.

Con esta configuraci�n, se van sustituyendo los cables conductores de diferentes longitudes y secciones y se van tomando medidas de $\Delta V$ e $I$.
