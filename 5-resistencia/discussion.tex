\section{Cuestiones}\label{sec:discusion}

\subsection{}

Se dice que la ley de Ohm es una ley emp�rica porque est� basada en observaciones experimentales
y no por haber sido derivada de principios te�ricos fundamentales.

En concreto, la ley tiene una base experimental.
En la d�cada de 1820, Georg Simon Ohm realiz� una serie de experimentos en los que midi� la relaci�n
entre voltaje, corriente y resistencia para varios materiales, observ� patrones consistentes y formul�
su ley bas�ndose en esos resultados.

Por esta raz�n, la ley de Ohm solo es v�lida bajo ciertas condiciones.
Por ejemplo, la ley se aplica a materiales �hmicos, pero no a materiales no �hmicos en los que
la resistencia depende de la corriente.

Por �ltimo, no es una ley fundamental, en el sentido de que no ha sido derivada de
principios fundamentales de f�sica, como s� que lo fueron, por ejemplo, las ecuaciones del movimiento de Newton.


