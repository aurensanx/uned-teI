\section{Conclusiones}

En esta pr�ctica hemos obtenido la constante el�stica de 3 muelles distintos mediante medidas de elongaci�n y del periodo del movimiento oscilatorio.

Las medidas tomadas para el periodo en los muelles 2 y 3 tienen menos dispersi�n que en el muelle 1, ya que adquirimos m�s destreza a la hora de tomar las
mediciones con la pr�ctica.
Por esta raz�n, los errores absoluto y relativo de los muelles 2 y 3 son menores son menores que para el muelle 1.

Adem�s, para el muelle 3 la masa a�adida fue menor que para los muelles 1 y 2. Esta puede ser la raz�n por la que el valor $\alpha$ s�
est� dentro del rango te�rico, ya que las oscilaciones del movimiento fueron m�s peque�as y la aproximaci�n
del movimiento a un movimiento arm�nico m�s realista.

