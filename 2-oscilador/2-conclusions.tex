\section{Conclusiones}

En esta pr�ctica hemos obtenido la constante el�stica de 3 muelles distintos mediante dos m�todos distintos.

Las medidas tomadas para el periodo del movimiento arm�nico en los muelles 2 y 3 tienen menos dispersi�n que en el muelle 1.
Por esta raz�n, las medidas de error absoluto y relativo son menores para los muelles 2 y 3.

Adem�s, para el muelle 3 la masa a�adida fue menor que para los muelles 1 y 2. Esta puede ser la raz�n por la que el valor $\alpha$ s�
est� dentro del rango te�rico, porque las oscilaciones del movimiento arm�nico fueron m�s peque�as y la aproximaci�n
del movimiento a un movimiento arm�nico m�s realista.

