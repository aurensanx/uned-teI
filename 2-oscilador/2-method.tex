\section{Dispositivo experimental}


Para comprobar la ley de Hooke y determinar la constante el�stica de los muelles, pesamos por separado el muelle y el portamasas.
Colocamos el muelle en el soporte y colgamos de �l el portamasas. En este punto, a�adimos una masa suficiente al portamasas
para que el muelle quede tenso.

A la suma de estas 3 masas, la del muelle, del portamasas y la masa inicial para tensar el muelle, la denominamos $m_0$.

Con el sistema en equilibrio, medimos la longitud del muelle, y asignamos el valor a la variable $y_0$.

Ya estamos preparados para a�adir distintas masas $m$ al portamasas y anotar las distintas longitudes $y$. La figura~\ref{fig:compression}
muestra las dos situaciones que hemos mencionado.

\begin{figure}
    \begin{tikzpicture}[black!75]

        % Supporting structure
        \fill [pattern = north west lines] (-1.5,0) rectangle ++(3,.2);
        \draw[thick] (-1.5,0) -- ++(3,0);

        % Spring + Arrows
        \draw[] (0,0) -- ++(0,-0.25);
        \draw[decoration={aspect=0.3, segment length=1.2mm, amplitude=2mm,coil},decorate] (0,-0.25) -- ++(0,-2.25) node[midway,right=0.25cm,black]{$k$};
        \draw[] (0,-2.5) -- ++(0,-0.3) node[coordinate](c1){} node[draw,fill=grey!30,minimum width=1cm,minimum height=0.5cm,anchor=north,label=east:$m_0$](M){};;

        \begin{scope}[xshift=4cm]
            % Supporting structure
            \fill [pattern = north west lines] (-1.5,0) rectangle ++(3,.2);
            \draw[thick] (-1.5,0) -- ++(3,0);

            % Spring + Arrows
            \draw[] (0,0) -- ++(0,-0.25);
            \draw[decoration={aspect=0.3, segment length=1.4mm, amplitude=2mm,coil},decorate] (0,-0.25) -- ++(0,-2.75) node[midway,right=0.25cm,black]{$k$};
            \draw[] (0,-3) -- ++(0,-0.3)node[coordinate](c2){} node[draw,fill=grey!30,minimum width=1cm,minimum height=0.5cm,anchor=north,label=east:$m + m_0$](M){};
        \end{scope}


        \draw[dashed,gray] (-0.5,-3.3) -- ++(-0.75,0)coordinate(c22);
        \draw[dashed,gray] (3.5, -3.8) -- ++(-0.75,0) coordinate(c12);
        \draw[latex-latex] (2.75, -3.8)-- (2.75,0)node[midway,left]{\small $y$};
%
        \draw[dashed,gray] (0,0) -- ++(-1.5,0) coordinate(c23);
        \draw[latex-latex] (-1.25,0)-- ++(0,-3.3)node[midway,left]{\small $y_0$};
    \end{tikzpicture}
    \caption{Situaci�n inicial del sistema masa-muelle y situaci�n de medici�n para la masa $m$.}
    \label{fig:compression}
\end{figure}

// TODO movimiento oscilatorio arm�nico
