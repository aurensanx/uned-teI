\section{Dispositivo experimental}


En primer lugar, para determinar la constante el�stica de un muelle por el m�todo est�tico, pesamos por separado
en una balanza el muelle y un portamasas que nos servir� para a�adir m�s adelante distintas masas.

La balanza tiene una precisi�n de 1\,g.

A continuaci�n, colocamos el muelle en el soporte, colgamos de �l el portamasas y a�adimos una masa suficiente para que el muelle quede tenso.

La suma de estas 3 masas (muelle, portamasas y masa inicial para tensar el muelle), ser� $m_0$.

Con el sistema en equilibrio, medimos con una regla la longitud del muelle, desde el gancho superior del soporte hasta la
superficie inferior del portamasas. Este valor ser� $y_0$.

La precisi�n de la regla usada es 1\,mm.

En este punto, ya estamos preparados para a�adir distintas masas $m$ al portamasas y anotar las distintas longitudes $y$. En la figura~\ref{fig:compression}
se muestra un esquema de la situaci�n inicial, y tras a�adir una masa $m$ al sistema.

\begin{figure}
    \begin{tikzpicture}[black!75]

        % Supporting structure
        \fill [pattern = north west lines] (-1.5,0) rectangle ++(3,.2);
        \draw[thick] (-1.5,0) -- ++(3,0);

        % Spring + Arrows
        \draw[] (0,0) -- ++(0,-0.25);
        \draw[decoration={aspect=0.3, segment length=1.2mm, amplitude=2mm,coil},decorate] (0,-0.25) -- ++(0,-2.25) node[midway,right=0.25cm,black]{$k$};
        \draw[] (0,-2.5) -- ++(0,-0.3) node[coordinate](c1){} node[draw,fill=grey!30,minimum width=1cm,minimum height=0.5cm,anchor=north,label=east:$m_0$](M){};;

        \begin{scope}[xshift=4cm]
            % Supporting structure
            \fill [pattern = north west lines] (-1.5,0) rectangle ++(3,.2);
            \draw[thick] (-1.5,0) -- ++(3,0);

            % Spring + Arrows
            \draw[] (0,0) -- ++(0,-0.25);
            \draw[decoration={aspect=0.3, segment length=1.4mm, amplitude=2mm,coil},decorate] (0,-0.25) -- ++(0,-2.75) node[midway,right=0.25cm,black]{$k$};
            \draw[] (0,-3) -- ++(0,-0.3)node[coordinate](c2){} node[draw,fill=grey!30,minimum width=1cm,minimum height=0.5cm,anchor=north,label=east:$m + m_0$](M){};
        \end{scope}


        \draw[dashed,gray] (-0.5,-3.3) -- ++(-0.75,0)coordinate(c22);
        \draw[dashed,gray] (3.5, -3.8) -- ++(-0.75,0) coordinate(c12);
        \draw[latex-latex] (2.75, -3.8)-- (2.75,0)node[midway,left]{\small $y$};
%
        \draw[dashed,gray] (0,0) -- ++(-1.5,0) coordinate(c23);
        \draw[latex-latex] (-1.25,0)-- ++(0,-3.3)node[midway,left]{\small $y_0$};
    \end{tikzpicture}
    \caption{Situaci�n inicial del sistema masa-muelle y situaci�n de medici�n para la masa $m$.}
    \label{fig:compression}
\end{figure}

Para tomar las mediciones del periodo del movimiento oscilatorio, se retira la regla, se desplaza el muelle hacia abajo ejerciendo una ligera fuerza, y a
continuaci�n se suelta.

Se dejan pasar unas oscilaciones hasta que el movimiento se estabilice y se cuenta con un cron�metro el tiempo $t$ que tardan en completarse $n$ oscilaciones.

De este modo, al dividir el tiempo total $t$ por $n$ oscilaciones para calcular el periodo $T$, se minimiza el error debido al tiempo de reacci�n
para operar el cron�metro, y tambi�n el error debido a la subjetividad a la hora de reconocer el punto de inicio de una oscilaci�n.

A su vez, se toman 3 medidas distintas del tiempo $t$. Calcularemos la media y la dispersi�n de las 3 medidas, lo que nos proporcionar�
m�s informaci�n del error en la toma de las mediciones temporales.

La precisi�n del cron�metro usado es de $0.01$\,s.
