\section{Discusi�n}
%%%%%%%%%%%%%%%%%%%%

Las cuatro partes de la memoria son importantes: la introducci�n para
plantear el problema, la metodolog�a para explicar el plan de trabajo
desarrollado, los resultados para mostrar lo obtenido con la metodolog�a
y la discusi�n para analizar la relaci�n entre las tres partes anteriores
y con el objetivo del trabajo.

Los tiempos empleados para llegar al resultado obtenido son peque�os,
comparados con los habituales siguiendo otras metodolog�as que favorecen
la redundancia de la informaci�n (v�ase, por ejemplo \cite{IUED2010}).
Al incluir una referencia, ya no se trata de una opini�n: se supone
que quien lo ha escrito ha avalado dicha afirmaci�n (bas�ndose, puede
ser, en su experiencia).

Adem�s, como resultado tambi�n se ha obtenido un documento f�cilmente
reutilizable, en \LaTeX{} que, por ejemplo, puede ser incluido en
un cap�tulo de un libro sin m�s que cambiar la clase de documento
\cite{Lamport1994}. Esto no estaba dentro de los objetivos, ni la
metodolog�a se indic� para ello, pero es algo nuevo que hay que destacar.

No s�lo se deben discutir los logros, sino tambi�n las limitaciones.
Por ejemplo, la limitaci�n de este m�todo de escribir trabajos es
que requiere un tiempo muy grande para completar los detalles de la
bibliograf�a y la instrumentaci�n. Se propone recopilar esta informaci�n
de los trabajos sucesivos y reutilizarla en los futuros, para ahorrar
tiempo.

