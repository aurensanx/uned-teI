\section{Cuestiones}\label{sec:discusion}

\subsection{Expresiones de la constante el�stica $k$ y energ�a potencial $U$}

Para obtener la expresi�n de $k$ partimos de la Ley de Hooke:

\begin{equation}
    \label{eq:1}
    F = - k \, \Delta x
\end{equation}

El signo menos solos nos indica que $F$ es una fuerza recuperadora que se opone al movimiento, as� que podemos
obviarlo para nuestros c�lculos.

Necesitamos dos definiciones adicionales. El esfuerzo $\sigma$ es la fuerza por unidad de �rea:

\begin{equation}
    \label{eq:2}
    \sigma = \frac{F}{A}
\end{equation}

La deformaci�n $\epsilon$ es el alargamiento entre la longitud inicial del sistema:

\begin{equation}
    \label{eq:3}
    \epsilon = \frac{\Delta x}{L}
\end{equation}

Para materiales el�sticos, el m�dulo de Young o de elasticidad $E$ es la relaci�n entre el esfuerzo $\sigma$ y la deformaci�n $\epsilon$:

\begin{equation}
    \label{eq:4}
    E = \frac{\sigma}{\epsilon}
\end{equation}

Sustituyendo las expresiones \ref{eq:2}, \ref{eq:3} y \ref{eq:4} en \ref{eq:1}, obtenemos:

\begin{equation}
    \label{eq:5}
    k = \frac{F}{\Delta x} = \frac{A \, \sigma}{L \, \epsilon} = \frac{A \, E}{L}
\end{equation}

Por otro lado, para obtener la expresi�n de la energ�a potencial $U$, utilizamos la ley de la conservaci�n de la energ�a.

El cambio de la energ�a potencial del sistema entre la posici�n de equilibrio y $\Delta x$ es igual al trabajo $W$ necesario realizado sobre
el sistema para alargarlo hasta la posici�n $\Delta x$. La fuerza requerida cambia con la distancia,
de manera que se calcula con la integral:

\begin{equation}
    \label{eq:6}
    W = \int_{0}^{\Delta x} k \,  x \,  d x = \frac{1}{2} \, k \, \Delta x^2
\end{equation}

\subsection{C�lculo de $k$}

La fuerza que se ejerce sobre el resorte es igual a $m \, g$. En la posici�n de equilibro, esta fuerza se iguala a la fuerza que ejerce
el resorte en el sentido contrario, que es igual a $k \, \Delta x$. Por tanto, tenemos:

\begin{equation*}
    \label{eq:7}
    k = \frac{m \, g}{\Delta x} = \frac{80 \, \text{kg} \cdot 9.81 \, \text{m/s$^2$}}{4 \times 10^{-3} \, \text{m}} = 196.2 \, \text{kN/m}
\end{equation*}

\subsection{M�dulo de elasticidad $E$}

Utilizando la expresi�n obtenida en \ref{eq:5}, despejamos $E$:

\begin{equation*}
    E = \frac{k\,L}{A} = \frac{196.2 \, \text{kN/m} \cdot 0.11 \text{m}}{\pi \, (1.25 \times 10^{-2} \text{m})^2} = 4.4 \times 10^{7} \, \text{N/m$^2$}
\end{equation*}

\subsection{Energ�a del sistema}

Si a partir del momento en el que una fuerza externa var�a la energ�a potencial del muelle ya no existen fuerzas externas al sistema masa-muelle, la energ�a total
del sistema se conservar�.

En ese caso, la energ�a se transformar� peri�dicamente entre energ�a potencial, en un extremo del movimiento, a energ�a cin�tica, en la posici�n natural del muelle.
La energ�a total ser� una combinaci�n de ambas en cualquier otro punto.

Si consideramos que existen fuerzas de rozamiento, por ejemplo, con el aire, o cualquier otro tipo de fuerza disipadora de energ�a, se puede considerar
que el sistema masa-muelle no est� aislado y en ese caso la energ�a total del sistema no se conservar�, sino que ir� disminuyendo con el tiempo.

\subsection{M�dulo de Young experimental}

// TODO valores experimentales

El m�dulo de Young de los muelles no es el mismo m�dulo de Young del material del que est� hecho el muelle.

La capacidad de un muelle para deformarse ante una fuerza y volver a su posici�n original cuando cesa esta fuerza depende otros factores adem�s del
material del que est� hecho, y es distinto a la capacidad que tendr�a una varilla recta del material.

