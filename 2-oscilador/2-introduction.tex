\section{Introducci�n}
%%%%%%%%%%%%%%%%%%%%%%%

La memoria de un trabajo cient�fico consiste en el planteamiento de
un problema (de modo que quien lo lea sepa lo que est� hecho y qu�
se est� resolviendo, esto es, el objetivo del trabajo), la explicaci�n
de los pasos dados para resolverlo (de modo que quien lo lea lo pueda
reproducir por s� mismo), la exposici�n de los resultados obtenidos,
la discusi�n cr�tica de los mismos (que es demostrar por qu� los
resultados son los que son, cu�n fiables son, y el significado que
tienen para la soluci�n buscada del problema), y unas conclusiones
que aclaren hasta d�nde se ha llegado en la soluci�n del problema
inicialmente planteado \cite{May2006}.


El objetivo de esta plantilla es servir de punto de partida (tanto
en estructura como en estilo) para la elaboraci�n de memorias de trabajos
para la asignatura de F�sica Computacional I.


%%%%%%%%%%%%%%%%%%%%%%%%%%%%%

