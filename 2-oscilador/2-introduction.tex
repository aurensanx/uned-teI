\section{Introducci�n}

El sistema masa-muelle es un caso sencillo que nos permite estudiar la fuerza el�stica y tambi�n el movimiento oscilatorio arm�nico.

En primer lugar, utilizaremos la ley de Hooke para determinar la constante el�stica de distintos muelles mediante el llamado m�todo est�tico.

Para ello, utilizaremos la relaci�n:

\begin{equation}
    k (y - y_0) = mg
\end{equation}

En esta expresi�n, $y_0$ es una longitud de referencia, que corresponde a la longitud del muelle cuando est� en posici�n de equilibrio y
sometido al peso de una masa $m_0$.
La longitud $y$
es la longitud del muelle cuando a la masa $m_0$ se ha a�adido una masa $m$.

Mediante mediciones de las longitudes $y$ para distintas masas $m$ podemos determinar la constante
el�stica $k$ de cada muelle.


