\section{Introducci�n}

El sistema masa-muelle es un caso sencillo que nos permite estudiar la fuerza el�stica y tambi�n el movimiento oscilatorio arm�nico.

En primer lugar, utilizaremos la ley de Hooke para determinar la constante el�stica de los muelles, utilizando la relaci�n:
\begin{equation}
    k \, (y - y_0) = m \,g
\end{equation}

En esta expresi�n, $y_0$ es una longitud de referencia, que corresponde a la longitud del muelle cuando est� en posici�n de equilibrio y
sometido al peso de una masa inicial $m_0$.
La longitud $y$
es la longitud del muelle cuando a la masa $m_0$ se ha a�adido una masa $m$.

Mediante mediciones de las longitudes $y$ para distintas masas $m$ podemos determinar la constante
el�stica $k$ de cada muelle.
Este es el llamado m�todo est�tico de obtenci�n de la constante el�stica.


En segundo lugar, desplazando ligeramente el sistema masa-muelle de su posici�n de equilibrio y liber�ndolo posteriormente,
podemos generar un movimiento oscilatorio arm�nico,
a partir del cual se puede calcular de nuevo la constante el�stica del muelle.
Este es el m�todo din�mico.

La justificaci�n te�rica es la siguiente. La ecuaci�n de movimiento del sistema una vez se ha desplazado de la posici�n de equilibrio es:
\begin{equation}
    \label{eq:1.1}
    M \, \frac{d^2y}{dt^2} + k \, y = 0
\end{equation}
La ecuaci�n \ref{eq:1.1} describe un movimiento oscilatorio, con periodo de oscilaci�n:
\begin{equation}
    T = 2 \pi \sqrt {\frac{M}{k}}
\end{equation}
Conocida la masa $M$ y midiendo el periodo de las oscilaciones, se obtiene la constante el�stica $k$.

En este movimiento, la masa $M$ est� compuesta por la masa $m$ que se cuelga del muelle, la masa inicial $m_0$ y la masa efectiva del propio muelle, $m_{ef}$.

La masa efectiva $m_{ef}$ del muelle es una fracci�n de su masa total, ya que en el movimiento oscilatorio no se mueve el muelle por completo, sino solo
una parte.
