\appendix


\section{Ap�ndice}
%%%%%%%%%%%%%%%%%%%

En un ap�ndice se recoge la informaci�n que puede ser necesaria para
entender la memoria y que, por lo espec�fica, puede ser dif�cil de
encontrar en otro lugar. Tambi�n se pueden incluir en un ap�ndice
los c�lculos muy largos que desviar�an la atenci�n de la exposici�n
en el cuerpo de la memoria. Por ejemplo, ecuaciones en l�nea, $y=ax+b$,
desplegadas
%
\[
    e^{i\pi}+1=0
\]
%
numeradas
%
\begin{equation}
    \label{eq:interconversion}
    E=mc^2
\end{equation}
%
o en varias l�neas
%
\begin{eqnarray}
    \label{eq:schrodinger}
    i\hbar\frac{\partial}{\partial t}\Psi(x,t) & = & H\Psi(x,t)\nonumber \\
    & = & -\frac{\hbar^{2}}{2m}\frac{\partial^{2}}{\partial x^{2}}\Psi(x,t)+U(x)\Psi(x,t)\nonumber \\
    & = & -\frac{\hbar^{2}}{2m}\frac{\partial^{2}}{\partial x^{2}}\Psi(x,t)+\frac{1}{2}kx^{2}\Psi(x,t)
\end{eqnarray}
%
o listados de programa como
%
\lstinputlisting[language=C]{helloworld0.c}

