\section{Introducci�n}

El pol�metro es un instrumento que sirve para medir varias magnitudes: voltaje, corriente y resistencia.
Tiene distintas escalas, y estas marcan el posible rango de valores que el aparato permite medir.

La resistencia normalmente solo se puede medir en corriente continua (c.c.), mientras que el voltaje y la corriente tambi�n
se pueden medir en corriente en alterna (c.a.).

El pol�metro usado en la pr�ctica pose�a 4 entradas.
La primera era una entrada com�n para unir al potencial m�s bajo (tierra).
La segunda era una entrada de car�cter general, con la que se miden voltajes y resistencias.
La tercera y cuarta eran entradas de corriente, con distinta escala.

Por su parte, el osciloscopio es un instrumento que permite medir se�ales el�ctricas.
Su uso principal involucra se�ales dependientes del tiempo.

Mediante el uso de un tubo de rayos cat�dicos (TRC) y varios componentes adicionales, el osciloscopio emite un haz de electrones
que choca contra una pantalla fosforescente.
De este modo, se puede observar y analizar en la pantalla la se�al el�ctrica aplicada.

El osciloscopio usado en la pr�ctica dispon�a de un gran n�mero de dispositivos de control y ajuste.
Entre ellos, destacan el control de intensidad y foco, control de posici�n vertical, control continuo de la base de tiempos,
entradas de dos canales y mandos de atenuaci�n de entrada.




